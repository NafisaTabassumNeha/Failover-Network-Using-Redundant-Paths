\documentclass[12pt]{report}
\usepackage[utf8]{inputenc}
\usepackage[T1]{fontenc}
\usepackage{fontspec}
\usepackage{natbib}
\setmainfont{Times New Roman}
\newfontfamily\hackfont{Hack}
\usepackage{titlesec}
\usepackage{fancyhdr}
\usepackage{fancyvrb}
\usepackage{listings}
\usepackage{xcolor}
\usepackage{graphicx}
\usepackage{longtable}
\usepackage{caption}
\usepackage{hyperref}
\usepackage{setspace}
\usepackage{geometry}

\geometry{a4paper, margin=1in}
\renewcommand{\contentsname}{Table of Contents}
\hypersetup{colorlinks=true, linkcolor=black, urlcolor=blue}
  

% Header/Footer setup
\pagestyle{fancy}
\fancyhf{} % clear all header and footer fields
\fancyhead[L]{Chapter \thechapter} % Left header: Chapter number
\fancyhead[R]{\nouppercase{\leftmark}} % Right header: Chapter title
\fancyfoot[C]{\thepage} % Center footer: page number

% Make sure chapter marks are updated
\renewcommand{\chaptermark}[1]{\markboth{#1}{}}

% Prevent header on chapter first page
\fancypagestyle{plain}{
  \fancyhf{}
  \fancyfoot[C]{\thepage}
  \renewcommand{\headrulewidth}{0pt}
}

\begin{document}

\begin{titlepage}					% Cover Start
\begin{center}
    \includegraphics[width=0.35\textwidth]{pictures/logo.png} \\
    \vspace{0.5cm}
    \textbf{\Huge \textcolor{teal}{NOTRE DAME} \textcolor{teal}{UNIVERSITY}} \\
    \vspace{0.5cm}
    \textbf{\Huge \textcolor{orange}{BANGLADESH}} \\
    
    \vspace{0.9cm}
    \textbf{\Huge \underline{Computer Networks Project Report}} \\
    
    \vspace{1.5em}
    \begin{flushleft}
    	\textbf{\Large Course Code: CSE-3204} \\
		\vspace{0.3cm}        
        \textbf{\Large Course Title: Computer Networks Lab} \\
        \vspace{0.3cm}
        \textbf{\Large Project Title: Failover Network Using Redundant Paths} \\
		\vspace{0.3cm}        
        \textbf{\Large Batch : CSE (19+20)}        
    \end{flushleft}
    
    \vspace{1em}
    \begin{flushleft}
        \textbf{\Huge \textcolor{blue}{Submitted by:}} \\
        \vspace{0.3cm}
        \textbf{\Large Sazzad Jelani \hspace{3.9cm} ID: 0692220005101003} \\
        \vspace{0.3cm}
        \textbf{\Large Nafisa Tabassum \hspace{3cm} ID: 0692220005101008} \\
		\vspace{0.3cm}
		\textbf{\Large Sadia Islam Mim \hspace{3cm} ID: 0692220005101010} \\
		\vspace{0.3cm}        
        \textbf{\Large Istiak Alam (CSE 20) \hspace{1.9cm} ID: 0692230005101005} \\
        \vspace{0.5cm}
        \textbf{\Large Submission Date: }{\Large \textbf{\today}\par}
    \end{flushleft}
    \vfill
    \begin{flushleft}
        \textbf{\Huge \textcolor{blue}{Submitted to:}} \\
        \vspace{0.3cm}
        \textbf{\Large Dr. Fernaz Narin Nur} \\
        \vspace{0.3cm}
        \textbf{\Large Adjunct Professor,} \\
        \vspace{0.3cm}
        \textbf{\Large Notre Dame University Bangladesh}
    \end{flushleft}
\end{center}
\end{titlepage}						% Cover End

\begin{abstract}
\thispagestyle{empty}
In modern enterprise and institutional networks, ensuring high availability and fault tolerance is critical for continuous service delivery. A single point of failure, especially in the default gateway or switching paths, can disrupt communication across the entire network. This project addresses this challenge by designing and implementing a failover network using redundant paths through the use of the Spanning Tree Protocol (STP) and the Hot Standby Router Protocol (HSRP), including its extended form—Multiple HSRP (MHSRP).\\

The goal is to build a network topology that remains resilient even in the event of hardware failure by automatically redirecting traffic through backup paths and routers without human intervention. The implementation includes VLAN segmentation for network efficiency, Inter-VLAN routing, STP to avoid loops, and MHSRP for seamless gateway redundancy. Practical testing and simulation were carried out in Cisco Packet Tracer, demonstrating successful failover transitions and recovery. This project not only proves the feasibility of fault-tolerant networking at scale but also prepares the groundwork for more advanced high-availability designs in the future.\\
\end{abstract}
\newpage

\tableofcontents
\thispagestyle{empty}
\clearpage
\pagenumbering{arabic}


\chapter{Introduction}

In today’s digital era, uninterrupted network connectivity is fundamental to the operation of any organization or institution. Businesses, educational institutions, and critical infrastructure all rely heavily on the availability and reliability of their networks. A single point of failure - such as the failure of a router or a switch - can result in serious communication breakdowns, service outages, and potentially huge losses in productivity and revenue.\\

To address such challenges, modern networks are designed with failover capabilities and redundancy protocols. Failover networks are structured in such a way that if one pathway or device fails, alternate routes or backup devices automatically take over, ensuring continuous connectivity. These redundant paths and protocols are essential for maintaining high availability, fault tolerance, and efficient data routing.\\

This project, titled \textbf{“Failover Network Using Redundant Paths”}, focuses on designing and implementing a reliable and resilient network infrastructure using Cisco Packet Tracer. The core idea is to deploy a combination of Layer 2 and Layer 3 redundancy techniques - specifically, the Spanning Tree Protocol (STP) for loop avoidance and the Hot Standby Router Protocol (HSRP) and its extension, Multiple HSRP (MHSRP), for gateway failover.\\

In the designed topology, each local area network (LAN) is connected through multiple routers and switches. Virtual LANs (VLANs) are used for logical segmentation, and trunk links are implemented to carry multiple VLAN traffic between devices. MHSRP allows both routers to serve as active and standby gateways for different VLANs, enhancing load balancing and reliability. If one router fails, the other automatically takes over the traffic for all VLANs, minimizing downtime and ensuring consistent network operation.\\

This chapter introduces the problem domain, outlines the motivation behind the project, and gives an overview of the approach taken. The following chapters delve into the technical implementation, configuration, challenges faced, and testing procedures used to validate the failover functionality.

\newpage

\chapter{Network Topology Overview}
\chapter{Problem Statement}
\section{Analyzing the Problem}
\section{Project Objective}
\chapter{Proposed Solution and Design Approach}
\chapter{Background Study}
\chapter{Routing Protocols}
\section{Spanning Tree Protocol}
\section{Hot Standby Router Protocol}
\section{Multiple Hot Standby Router Protocol}
\newpage

\chapter{HSRP Configuration}

\section{Router Configuration}
\subsection*{Step 1 Configure IP on Router1}
\begin{Verbatim}[fontsize=\small, formatcom=\hackfont\color{black}, frame=single]
enable
conf t
interface fa0/0
ip address 192.168.1.1 255.255.255.0
no shutdown
exit
\end{Verbatim}

\subsection*{Step 2 Configure IP on Router2}
\begin{Verbatim}[fontsize=\small, formatcom=\hackfont\color{black}, frame=single]
enable
conf t
interface fa0/0
ip address 192.168.1.2 255.255.255.0
no shutdown
exit
\end{Verbatim}

\subsection*{Step 3: Configure HSRP for Gateway Redundancy On Router1}
\begin{Verbatim}[fontsize=\small, formatcom=\hackfont\color{black}, frame=single]
conf t
interface fa0/0
standby 1 ip 192.168.1.2
standby 1 priority 110
standby 1 preempt
exit
\end{Verbatim}

\subsection*{Step 4: Configure HSRP for Gateway Redundancy On Router2}
\begin{Verbatim}[fontsize=\small, formatcom=\hackfont\color{black}, frame=single]
conf t
interface fa0/0
standby 1 ip 192.168.1.1
standby 1 priority 100
standby 1 preempt
exit
\end{Verbatim}

\section{PC IP Configuration}
\subsection*{Step 5: Configure PCs}
\begin{Verbatim}[fontsize=\small, formatcom=\hackfont\color{black}]
Set static IP manually:
PC1: 
IP Address: 192.168.1.10 
Subnet Mask: 255.255.255.0
Default Gateway: 192.168.1.1

PC2:
IP Address: 192.168.1.20
Subnet Mask: 255.255.255.0
Default Gateway: 192.168.1.1
\end{Verbatim}

\section{Switch Configuration}
\subsection*{Step 5: Enable Spanning Tree Protocol (Optional)}
On Switch1 and Switch2
\begin{Verbatim}[fontsize=\small, formatcom=\hackfont\color{black}, frame=single]
enable
conf t
spanning-tree vlan 1 priority 4096
exit
\end{Verbatim}

\newpage

\section{Testing in HSRP Protocol} 

\chapter{Multiple Hot Standby Router Protocol Configuration}
\section{Big Network Problems}
\section{Implementation of MHSRP in Network}
\section{Router Configuration of MHSRP}
\section{PC IP Configuration}
\section{Switch Configuration}
\section{Testing in MHSRP Protocol}
\chapter{Troubleshooting and Observations}
\chapter{Future Work and Improvements}
\chapter{References}

\end{document}